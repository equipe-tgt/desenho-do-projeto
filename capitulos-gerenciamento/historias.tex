\section{Histórias de usuário}
Para uma melhor estimativa e planejamento do projeto, histórias de usuário foram desenvolvidas para expressar funcionalidades do aplicativo que agreguem valor para seus usuários. Elas se originaram a partir das opiniões e sugestões de pessoas próximas que foram consultadas sobre o propósito do desenvolvimento, como familiares, amigos e principalmente os professores da disciplina. A tabela abaixo apresenta a descrição de todas as histórias de usuário geradas.

\begin{quadro}[H]
\caption{Histórias de usuário}
\resizebox{\textwidth}{!}{%
\begin{tabular}{|l|}
\hline
\multicolumn{1}{|c|}{\textbf{Histórias de usuário}}                                                                                           \\ \hline
Eu, como usuário, \\ gostaria de fazer cadastro no aplicativo de maneira básica e simples.                                                       \\ \hline
Eu, como usuário, \\ gostaria de fazer login utilizando e-mail e senha para que eu possa usar as funcionalidades do aplicativo.                  \\ \hline
Eu, como usuário, \\ gostaria de poder fazer reset de senha caso esqueça meu login.                                                              \\ \hline
Eu, como usuário, \\ gostaria de poder visualizar meus dados do aplicativo: e-mail, nome, senha.                                                 \\ \hline
Eu, como usuário, \\ gostaria de poder ver as pessoas que convidei para minhas listas, e gostaria de poder ver os convites que recebi.           \\ \hline
Eu, como usuário, \\ gostaria de poder ver minhas configurações de conta, como idioma e versão, por exemplo.                                     \\ \hline
Eu, como usuário, \\ gostaria de poder deslogar do aplicativo se necessário.                                                                     \\ \hline
Eu, como usuário, \\ gostaria de poder criar uma ou mais lista de compras.                                                                       \\ \hline
Eu, como usuário, \\ quero poder visualizar as listas que criei com um resumo de quantidade de itens e membros.                                  \\ \hline
Eu, como usuário, \\ quero poder visualizar os itens de uma lista que eu criei.                                                                  \\ \hline
Eu, como usuário, \\ quero poder compartilhar a lista que criei com outro membro.                                                                \\ \hline
Eu, como usuário, \\ quero poder visualizar as pessoas com quem compartilhei a lista.                                                            \\ \hline
Eu, como usuário, \\ quero poder remover um membro da minha lista.                                                                               \\ \hline
Eu, como usuário, \\ quero poder adicionar mais itens a minha lista.                                                                             \\ \hline
Eu, como usuário, \\ quero poder editar informações de um item na minha lista e adicionar comentários.                                           \\ \hline
Eu, como usuário, \\ quero poder adicionar um item caso não ache ele na lista de categorias.                                                     \\ \hline
Eu, como usuário, \\ quero poder adicionar um item a uma categoria através do código de barras.                                                  \\ \hline
Eu, como usuário, \\ quero poder iniciar uma compra no app para indicar que estou no mercado.                                                    \\ \hline
Eu, como usuário, \\ quando iniciar a compra, quero poder ver todos os itens, \\ de todas as listas, ou filtrar por apenas uma lista em específico. \\ \hline
Eu, como usuário, \\ quando iniciar a compra, quero poder ver o total gasto e salvar o que foi comprado.                                         \\ \hline
Eu, como usuário, \\ quero poder usar a mesma lista em várias compras.                                                                           \\ \hline
Eu, como usuário, \\ quero poder registrar o local onde a compra foi realizada.                                                                  \\ \hline
\end{tabular}%
}
\fonte{Os Autores}
\end{quadro}
                                                               
