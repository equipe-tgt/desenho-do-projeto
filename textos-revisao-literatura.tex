% ---
% Capitulo de revisão de literatura
% ---
\chapter{Revisão Bibliográfica}
Nesta seção, os tópicos conceituais e teóricos relacionados ao projeto a ser desenvolvido que irão auxiliar o entendimento do mesmo serão descritos de forma a esclarecer tanto as partes de negócio, como as técnicas. Dessa forma, a compreensão do projeto como todo será mais fácil para o leitor.

\section{Organização financeira}
De acordo com o dicionário Michaelis \cite{Michaelis}, a palavra organização significa preparação de um projeto, com definição de procedimentos e metas. Assim, pode se dizer que organização financeira trata-se de cuidar das finanças (podendo elas ser empresarial, familiar, ou pessoal) de modo a atingir um objetivo.

Quando se trata de finanças pessoais ou familiares, ter uma boa organização financeira significa conhecer quais e quanto são suas despesas e receitas mensais \cite{PlanejamentoFinanceiroFamiliar}. Segundo o mesmo artigo, é necessário planejar como, quanto e onde a despesa será feita, além de fazer levantamento de preços.

Pode-se dizer então que as compras, categorizadas como despesas fixas devido a sua presença frequente em ambientes pessoais e familiares, precisam ser planejadas e organizadas para impulsionar uma melhor organização financeira entre os envolvidos da compra a fim de satisfazer as suas necessidades. Em um certo grau, é determinado a partir da compra o relacionamento entre usuário e marca a partir dos produtos de uma compra, de modo que seja fidelizado e compre o produto por paixão em detrimento de outros fatores.

\section{Listas}
As listas de afazeres, comumente chamadas de \textit{to do list} em inglês, é uma lista de lembretes textuais, como, por exemplo, “Ir ao médico”, “Comprar caixa de 12 ovos”, entre outros \cite{Towel}. Essas listas são encontradas em qualquer âmbito, e auxiliam a lembrar de tarefas ou coisas importantes que precisam sofrer uma ação do indivíduo, por exemplo, comprar, completar, finalizar.

Seguindo o mesmo princípio, uma lista de compras se refere a uma lista contendo nomes de produtos nos quais um indivíduo deseja adquirir, podendo conter ou não um limite de gasto total. As listas, além de ser um método eficaz para se organizar e lembrar do que foi escrito, é considerada uma ótima maneira de evitar que alguém ceda à produtos nos quais não quer comprar por serem prejudiciais à saúde da pessoa \cite{GroceryList} ou por qualquer outro motivo.

Existem três categorias de compras \cite{ComprasNaoPlanejadas}, sendo que dois tipos podem ter a presença de uma lista de compras:
\begin{enumerate}
\item Completamente planejada, incluindo na lista de compras os produtos e suas respectivas marcas.
\item Parcialmente planejada, incluindo apenas os produtos a serem comprados, mas a marca deles serão decididas no ato da compra.
\end{enumerate}

As compras completamente planejadas são consideras com pouco envolvimento emocional do consumidor, enquanto as parcialmente planejadas, apesar de planejadas, podem ser manipuladas por algum critério exterior, como uma promoção \cite{ComprasNaoPlanejadas}. No aplicativo a ser desenvolvido, uma lista de compras pode ser tanto completamente planejada ou parcialmente planejada, segundo a necessidade do usuário que está utilizando a plataforma. Visto que a lista de compras existe, a compra se torna planejada.

Uma lista de compras é o planejamento de uma compra, buscando o melhor custo-benefício. Enquanto uma compra a partir das listas de compras é o ato de comprar alguns itens de algumas listas conscientemente do que está precisando e de quão vantajoso está sendo adquirir determinados itens, ponderando suas necessidades e vontades.

\section{Teoria do Comportamento do Consumidor}

Segundo Richard Thaler, ecônomo norte-americano, cada consumidor possui seu próprio comportamento que impactam no momento pré-compra, durante a compra e pós-compra e, para cada tipo de comportamento, será necessário tipos diferentes de atrair esse público (através do \textit{marketing}). 

Para isso, foram criadas cinco teorias do comportamento do consumidor, que tentam descrever tanto o consciente quanto o inconsciente do consumidor, aplicando conceitos de economia, sociologia e psicologia. \cite{TeoriasConsumidor}

\begin{enumerate}

	\item \underline{Teoria da Racionalidade Econômica}: o indivíduo está em busca do melhor custo-benefício, não estando presente valores de forte relacionamento com determinadas marcas de sua preferência.

	\item \underline{Teoria Comportamental}: o indivíduo está em busca do produto que consegue satisfazê-lo melhor, baseando-se no ambiente que está inserido.

	\item \underline{Teoria Psicanalítica}: o indivíduo idealiza os produtos a serem comprados e tenta satisfazer seus desejos (sendo inconscientes) e vincular-se às marcas.

	\item \underline{Teorias Sociais e Antropológicas}: o indivíduo consome itens não apenas para sobreviver e saciar suas vontades, mas também para se posicionar no ambiente social e cultural inserido.

	\item \underline{Teoria Cognitivista}: sendo a mais comum para a criação de \textit{marketings}, essa teoria defende que o consumo é proveniente de todos os fatores que o indivíduo está inserido (social, cultural, cognitivos) e, assim, o indivíduo consome através de todos esses fatores e, ocasionalmente, fatores situacionais.
	
\end{enumerate}

Com essas teorias, torna-se evidente que o ato de fazer compra não é simplesmente para suprir suas necessidades, mas também passa por vontades, é influenciado pelo contexto inserido, seus valores culturais, sua posição social, a busca de \textit{status} social, buscando geralmente a melhor qualidade pelo menor preço possível.

O projeto Lixt será capaz de registrar os itens consumidos pelo usuário (que, como dito, são escolhidos influenciadamente), e garante a análise e histórico de dados, tornando-se cada vez mais consciente do custo-benefício dos itens, bem como registro de todas as suas experiências de compras em diferentes mercados e com diferentes produtos.

\section{Aplicativo \textit{Mobile}}
Um aplicativo \textit{mobile} trata-se de uma aplicação de \textit{software} que é desenvolvida exclusivamente ou acessível por um dispositivo móvel, como celulares e \textit{smartphones}. O uso dos dispositivos móveis cresceu disparadamente nos últimos anos, e é o principal meio de acesso no Brasil desde 2017 \cite{Celular}.

Atualmente, existem dois sistemas operacionais que são mais utilizados nas plataformas \textit{mobile}: o \textit{Android} e o \textit{IOS}, tratados como concorrentes das marcas \textit{Google} e \textit{Apple}. Os aplicativos desenvolvidos para os dois sistemas operacionais são divididos em dois tipos:
\begin{itemize}
\item Aplicativos nativos.
\item Aplicativos híbridos.
\end{itemize}

Os aplicativos nativos são desenvolvidos utilizando o \gls{SDK} do sistema operacional em questão, e não pode ser executados em outros ambientes. Por exemplo, um aplicativo desenvolvido com o \gls{SDK} da empresa \textit{Apple} não pode ser executado em ambiente \textit{Android} e vice-versa \cite{MobileApps}.

Já os aplicativos híbridos são desenvolvidos utilizando tecnologias web, como \gls{HTML}, \gls{CSS} e \textit{Javascript}, e por isso, podem ser executados em qualquer sistema operacional móvel através de um \textit{container} nativo. Apesar de poder ser executado em diversos dispositivos através de apenas um código, os aplicativos híbridos	 possuem certas limitações ao acessar recursos nativos do dispositivo, como câmeras, microfones e sistemas internos de armazenamento e memória, porém, atualmente já existem soluções que facilitam a comunicação entre aplicativo híbrido e o dispositivo, como o \textit{React Native} \cite{MobileApps}.





