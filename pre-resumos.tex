% ---
% RESUMOS
% ---

% resumo em português
\setlength{\absparsep}{18pt} % ajusta o espaçamento dos parágrafos do resumo
\begin{resumo}

O presente documento tem como objetivo apresentar o desenho do projeto Lixt, solução tecnólogica
que visa facilitar a gestão de controle de gastos em mercado e gerenciamento das listas de compras individuais ou colaborativas. 
A plataforma dá autonomia para o usuário gerir listas, membros de uma lista, produtos, categorias dos produtos e compras. O Lixt ainda se propõe a oferecer a análise das compras realizadas e fornecer o histórico de compras para que o usuário consiga obter melhor controle sobre suas despesas de mercado. 
Para implementação da proposta utilizamos o banco de dados MySQL com back-end construído em Java aliado ao Spring e ao Hibernate. Na implementação mobile utilizamos Javascript com o framework React-Native. A hospedagem dos recursos
ficará na plataforma Amazon Web Services. Para gerenciarmos a equipe durante o projeto optamos pela utilização do scrum e do kanban. 

 \textbf{Palavras-chaves}: lista de compras. compras. organização financeira.
\end{resumo}

% resumo em inglês
\begin{resumo}[Abstract]
 \begin{otherlanguage*}{english}
This document aims to present the design of the Lixt project, a technological solution that aims to facilitate the management of expenses control in groceries and management of individual or collaborative shopping lists.
The platform eanbles the user to manage lists, list members, products, product categories and purchases. Lixt also proposes to offer an analysis of the purchases made and provide the purchase history so that the user can obtain better control over their groceries expenses.
To implement the proposal, we decided to use the MySQL database and to develop the back-end in Java together with Spring and Hibernate. In the mobile implementation it uses Javascript with the React-Native framework. We chose to host the resources on the Amazon Web Services platform. To manage the team during the project, we chose to use scrum and kanban.
   \vspace{\onelineskip}

   \noindent 
   \textbf{Keywords}: shopping list. groceries. financial organization.
 \end{otherlanguage*}
\end{resumo}