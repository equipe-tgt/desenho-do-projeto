\section{Manutenibilidade da aplicação}

É fundamental para o desenvolvimento do projeto, tanto o previsto
quanto em avanços posteriores, que a aplicação atinja um nível adequado
de qualidade, e para tanto certos requisitos de manutenibilidade devem
ser estabelecidos.  Estes requisitos permitem a

\subsection{\emph{Logs}}

Como forma de monitorar a aplicação em tempo de execução,
especialmente na camada de servidor, \emph{logs} serão usados para
registrar o estado dos objetos. A ferramenta Log4j será usada, uma vez
que os membros do time já tem mais familiaridade com ela. Esta
ferramenta perimite o registro em diversos níveis, como
\begin{itemize}
  \em
\item info
\item debug
\item warn
\item error
\end{itemize}
e pode ser configurada para que apenas os dois últimos
sejam registrados no ambiente de produção. Desta forma a cada ponto de
falha da aplicação um log de nível apropriado será colocado para que
problemas sejam rapidamente identificados, analisados e resolvidos.

\subsection{Integração Contínua}

Visando manter o serviço sempre atualizado para o usuário, a
ferramenta de integração contínua Jenkins foi selecionada para a
implantação da aplicação \gls{backend} em produção. Ela
permite, a partir do código no repositório git, a execução de testes,
o build e o \gls{deploy} para o ambiente de produção, automatizando tarefas
complicadas de se executar em máquinas remotas.

Para o \gls{frontend}, no entanto, não existem ferramentas de
integração contínua, uma vez que as imagens do aplicativo mobile
precisam ser aprovadas pelas lojas de aplicativo. Contudo, ainda é
possível usar ferramentas de CI para a execução de testes e
verificação da integridade do código a cada nova versão.

\subsection{\emph{Code Conventions}}

As convenções de código são acordos internos ao time que visam
estandartizar a forma como os diversos integrantes do time produzem
seus códigos.  Elas visam facilitar o entendimento mútuo entre os
integrantes do time, de modo o estilo de progamação seja indistiguível
e independente de seus autores.  Geralmente, as convenções de código
estabelecem estilos para se organizar o código textualmente, isto é,
dizem respeito a forma como nomes de variáveis são escolhidas e
comentários são posicionados, por exemplo.

As convenções adotadas são baseadas na especificação da
\citeauthor{Oracle1997}, de 1997. Esta é comumente usadas para o
desenvolvimento na linguagem Java, e muito próxima do padão adotado em
JavaScript, e vale destacar o seguintes pontos:
\begin{itemize}
\item Minimizar o uso de variáveis, funções e objetos globais.
\item As declarações globais estarão preferencialmente no início do arquivo.
\item Declarar as variáveis próximo do ponto onde elas serão inicializadas.
\item A indentação é de 4 espaços.
\item Linhas mais longas que 80 caracteres serão quebradas e
  indentadas a 8 espaços.
\item Pacotes e variáveis com nomes curtos, em \texttt{camelCase} e substantivos.
\item Classes e interfaces em \texttt{CamelCase} e substantivos.
\item Métodos em \texttt{camelCase} e verbos.
\item Constantes em \texttt{UPPER\_CASE}.
\end{itemize}

No entanto, especificamente para a linguagem Java,
\begin{itemize}
\item Classes e métodos devem ser documentados com um comentário na
  seguinte forma, uma vez que as IDEs reconhecem este formato e
  formatam o text na forma de \emph{pop-ups} quando o cursor está
  sobre uma referência a esta classe.
\begin{verbatim}
/**
 * Class ListService
 *
 * Implementar endpoints para as funcionalidades de lista.
 */
\end{verbatim}
\end{itemize}

\subsection{\emph{Designs Patterns}}

Como padrões de projetos, serão usados 3 padrões amplamente utilizados pela comunidade de desenvolvimento: (I) Clean Code, (II) SOLID e (III) Gang Of Four. Esses 3 padrões podem ser utilizados tanto no backend quanto no frontend.

\subsubsection{\emph{Clean Code}}

O Clean Code é um conjunto de boas práticas para melhorar o entendimento do código, de modo que seja mais fácil sua leitura. Segue abaixo uma lista das principais boas práticas:

\begin{itemize}
	\item Nome Significativos para variáveis, classes, métodos, atributos e objetos, evitando abreviaturas desnecessárias e nomes não recorrentemente utilizados, sendo passíveis de busca.
	\item Evitar números mágicos, recomendando-se utilizar enums e constantes para que seja mais compreensível ao analisar o código.
	\item Evitar booleanos de forma implícita.
	\item Evitar condicionais negativos (por exemplo, uma função com nome "naoDevoExecutar()"), visto que dificulta a compreensão de código.
	\item Encapsular condicionais.
	\item Nunca comentar o óbio no código.
	\item Utilizar funções pequenas, tanto para melhorar a leitura quanto para respeitar os padrões do SOLID.
\end{itemize}

\subsubsection{\emph{SOLID}}

O SOLID é um acrônimo de 5 príncipios da Programação Orientada a Objetos, sendo fundamental para o desenvolvimento e manutenção de software.

\begin{itemize}
	\item \underline{\textbf{S}ingle Responsiblity Principle}: Uma classe deve ter apenas um motivo para mudar.
	\item \underline{\textbf{O}pen-Closed Principle}: Uma classe deve estar aberta para extensão, e fechada para modificação, recomendando sempre utilizar a herança e não modificar o código fonte original.
	\item \underline{\textbf{L}iskov Substitution Principle}: Uma classe derivada deve ser substituível por sua classe base.
	\item \underline{\textbf{I}nterface Segregation Principle}: Utilizar muitas interfaces específicas é melhor do que uma interface genérica.
	\item \underline{\textbf{D}ependency Inversion Principle}: Dependa de abstrações e não de implementações.
\end{itemize}

\subsubsection{\emph{Gang Of Four}}

O GoF (ou Gang of Four) se refere aos profissionais que criaram os 23 padrões de projetos, nos quais seus conceitos serão utilizados no desenvolvimento do Lixt.

Segue abaixo uma breve apresentação dos principais padrões utilizados:

\begin{itemize}
	\item \underline{Builder}: Será utilizado para criar instâncias legíveis de objetos complexos.
	\item \underline{Facade}: Encapsula regras de negócios complexas.
	\item \underline{Observer}: Observa e é capaz de reagir às mudanças de estado de um objeto.
\end{itemize}

\subsection{Testes}

Testes são uma ferramenta fundamental para o desenvolvimento da
aplicação, uma vez que garenatem, em tempo de compilação, o
comportamento correto do aplicativo. Para além disso, testes tem um
papel de documentação, uma vez que encapsulam de forma breve o
comportamento esperado das classes e métodos produzidos, e podem ser
consultados em caso de dúvida quando ao uso destes. Este tipo de teste
é chamado de teste unitário, em oposição aos testes de integração, que
verificam o funcionamento da aplicação de ponta-a-ponta, isto é, a
partir de uma chamada a um endpoint, apenas os serviços externos são
mimetizados, garantindo o funcionamento correto de toda a aplicação.

Desta forma, a construção de testes, de ambos os tipos é de extrema
importância para a elaboração do projeto visando a sua
manutenibilidade, e será o primeiro passo de uma sprint, após o
planejamento, a construção de testes relevantes para a tarefa,
seguindo os princípios do TDD\cite{TDD}. Como as ferramentas de teste são
específicas de cada linguagem, cada camada da aplicação fará uso de
frameworks distintos.

O \gls{backend} será testado com o framework JUnit, fazendo uso da
biblioteca Mockito quando necessário simular comportamentos de objetos
que não são o alvo da suite de teste. Este framework ainda oferece
ferramentas para testar o banco de dados, ou melhor, testar a conexão
com o banco e verificar o comportamento das classes de acesso a ele.
Já os teste de \gls{frontend} serão feitos com a biblioteca Jest,
que auxilia a construção de testes unitários, e por se tratar de uma
\gls{GUI}, as interções de usuário devem ser simuladas também. Para tanto, a
blioteca React Testing Library será usada.

%%% Local Variables:
%%% mode: latex
%%% TeX-master: "../../desenho"
%%% End:
