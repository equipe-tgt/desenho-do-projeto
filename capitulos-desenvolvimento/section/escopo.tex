\section{Escopo do Projeto}

Lixt é um aplicativo para gerenciamento de listas de compras compartilhadas ou não.

O aplicativo vai seguir a dinâmica de uso abaixo:
\begin{enumerate}
	\item O usuário cria uma lista de compras e insere todos os itens antes da compra;
	\item O usuário inicia um carrinho de compras quando chega ao mercado, nesse momento ele tem a opção de importar para aquele carrinho os itens das listas que ele possui em aberto (que ainda não foram finalizadas), marcando quais listas ele deseja que sejam incluídas. Com o carrinho de compras ele poderá anotar os preços dos itens, riscá-los e ver o total gasto. Ao finalizar a compra as listas são atualizadas, e já aparecem riscados os itens que já foram comprados;
	\item Quando o usuário definir que uma lista não é mais relevante ele poderá deletar a lista ou desmarcar todos os itens, para reutilizar a lista.
\end{enumerate}

A seguir listamos as principais funcionalidades como uma lista de tópicos para facilitar a visualização de quais funcionalidades dependem de outras de forma hierárquica:

\begin{itemize}
	\item \textit{Login}:
		\begin{itemize}
			\item Criar conta;
			\item Redefinir senha;
		\end{itemize}
	\item Editar uma lista:
		\begin{itemize}
			\item Adicionar item:
				\begin{itemize}
					\item Definir nome;
					\item Definir quantidade;
					\item Definir unidade de medida (un. ml, L etc.);
					\item Definir medida;
					\item Adicionar uma categoria:
						\begin{itemize}
							\item Criar nova categoria;
						\end{itemize}
					\item Adicionar um comentário;
					\item Atribuir a um usuário (caso a lista tenha sido compartilhada e pelo menos um convite já tenha sido aceito);
				\end{itemize}
			\item Remover itens;
			\item Convidar pessoas para a lista (apenas quem criou a lista):
				\begin{itemize}
					\item Enviar convite:
						\begin{itemize}
							\item Acompanhar \textit{status} (aceito ou pendente);
							\item Remover convite;
						\end{itemize}
				\end{itemize}
			\item Deletar uma lista;
			\item Limpar uma lista;
		\end{itemize}
	\item Iniciar um carrinho de compras:
		\begin{itemize}
			\item Selecionar listas para compor o carrinho;
			\item Informar o mercado onde a compra será realizada (automaticamente através da localização, se não estiver habilitada será solicitado que o usuário insira o nome do mercado);
			\item Exibir o valor total do carrinho;
			\item Informar a quantidade que será efetivamente comprada naquele momento (o usuário pode ter planejado 10 unidades e apenas comprar 5 naquele momento);
			\item Riscar itens;
			\item Finalizar um carrinho de compras;
		\end{itemize}
	\item Ver estatísticas:
		\begin{itemize}
			\item Selecionar uma lista e ver o total gasto naquela lista ao longo do tempo em um gráfico de linha:
				\begin{itemize}
					\item Selecionar um dos pontos do gráfico e ver detalhes daquela lista;
				\end{itemize}
			\item Selecionar uma lista para ver um gráfico de pizza com os valores médios gastos por categorias naquela lista;
			\item Verificar histórico de preços de um item (tabela com nome do produto, quantidade, marca, preço, mercado e data da compra):
				\begin{itemize}
					\item Selecionar uma lista, dentro da lista selecionada selecionar o produto para ver o histórico;
				\end{itemize}
		\end{itemize}
\end{itemize}

Para o \textit{Minimum Viable Product} (\label{sig:mvp}\hyperlink{s:mvp}{MVP}) vamos implementar as seguintes funcionalidades, as demais ficarão para o próximo semestre:

\begin{itemize}
	\item \textit{Login}:
		\begin{itemize}
			\item Criar conta;
			\item Redefinir senha;
		\end{itemize}
	\item Criar lista de compra:
		\begin{itemize}
			\item Atribuir um nome;
			\item Atribuir uma descrição;
			\item Importar uma lista anterior;
		\end{itemize}
	\item Editar uma lista:
		\begin{itemize}
			\item Adicionar item:
				\begin{itemize}
					\item Definir nome;
					\item Definir quantidade;
					\item Definir unidade de medida (un., ml., L etc);
					\item Definir medida;
					\item Adicionar a uma categoria:
						\begin{itemize}
							\item Criar uma nova categoria;
						\end{itemize}
					\item Adicionar um comentário;
					\item Atribuir a um usuário (caso a lista tenha sido compartilhada e pelo menos um convite já tenha sido aceito);
				\end{itemize}
			\item Remover itens;
			\item Convidar pessoas para a lista (apenas quem criou a lista):
				\begin{itemize}
					\item Enviar convite:
						\begin{itemize}
							\item Acompanhar  status (aceito ou pendente);
							\item Remover convite;
						\end{itemize}
				\end{itemize}
			\item Deletar uma lista;
			\item Limpar uma lista;
		\end{itemize}
	\item Iniciar um carrinho de compras:
		\begin{itemize}
			\item Selecionar listas para compor o carrinho;
			\item Informar o mercado onde a compra será realizada (será solicitado que o usuário insira o nome do mercado manualmente);
			\item Exibir o valor total do carrinho;
			\item Informar a quantidade que será efetivamente comprada naquele momento (o usuário pode ter planejado 10 unidades e apenas comprar 5 naquele momento);
			\item Riscar itens;
			\item Finalizar o carrinho de compras.
		\end{itemize}
\end{itemize}

\subsubsection{Requisitos Funcionais}

Os requisitos funcionais dizem respeito às funcionalidade que o sistema deve ter. O Quadro \ref{reqFuncionais} lista os requisitos funcionais, suas dependências, a sigla e a prioridade de implementação.

\begin{quadro}[H]
\caption{Requisitos funcionais}
\begin{tabular}{|l|l|l|l|l}
\cline{1-4}
\textbf{Sigla} & \textbf{Descrição}                                                                                                                                                                                                                                               & \textbf{Prioridade} & \textbf{Dependências} &  \\ \cline{1-4}
RF01           & \begin{tabular}[c]{@{}l@{}}\textit{Login}: o usuário deve ser capaz de criar \\ sua conta no aplicativo, definir sua senha e \\ realizar o \textit{login} no sistema.\end{tabular}                                                                                                 & Alta                &                       &  \\ \cline{1-4}
RF02           & \begin{tabular}[c]{@{}l@{}}O sistema deve possibilitar que o usuário \\ crie suas listas de compras e possa atribuir um\\ nome, uma descrição e ter a opção de importar \\ uma lista existente.\end{tabular}                                                     & Alta                & RF01                  &  \\ \cline{1-4}
RF03           & \begin{tabular}[c]{@{}l@{}}Editar uma lista: Possibilita ao usuário o \\ gerenciamento dos itens da lista, como adicionar \\ itens, remover e enviar convites para a lista.\end{tabular}                                                                         & Alta                & RF02                  &  \\ \cline{1-4}
RF04           & \begin{tabular}[c]{@{}l@{}}Iniciar um carrinho de compras: permitir que o \\ usuário importe várias listas de compras, informe \\ o local da compra, o total gasto, quantidade de \\ itens a ser comprados, riscar itens e finalizar \\ o carrinho.\end{tabular} & Alta                & RF03                  &  \\ \cline{1-4}
RF05           & \begin{tabular}[c]{@{}l@{}}Ver estatísticas: ver o histórico de valores \\ pagos em uma lista ao longo do tempo, ver \\ os valores gastos por categorias em uma lista, \\ ver o histórico de preços de um determinado \\ item ao longo do tempo.\end{tabular}    & Média               & RF04                  &  \\ \cline{1-4}
\end{tabular}
\fonte{Os Autores}
\label{reqFuncionais}
\end{quadro}


\subsubsection{Requisitos Não Funcionais}

De maneira simplificada, os requisitos não funcionais não estão relacionados diretamente às funcionalidades do sistema, mas ao seu funcionamento de um modo geral, ou seja, como ele as funcionalidade serão executadas.

No Quadro \ref{reqNaoFuncionais} estão elencados os requisitos não funcionais, cada um com sua nomenclatura, categoria e descrição.

\begin{quadro}[H]
\caption{Requisitos Não funcionais}
\begin{tabular}{llll}
\cline{1-3}
\multicolumn{1}{|l|}{\textbf{Nomenclatura}} & \multicolumn{1}{l|}{\textbf{Descrição}}                                                                                                                                                                                                            & \multicolumn{1}{l|}{\textbf{Categoria}} &  \\ \cline{1-3}
\multicolumn{1}{|l|}{RNF01}                 & \multicolumn{1}{l|}{\begin{tabular}[c]{@{}l@{}}Criptografia das senhas: Por uma questão \\ de segurança as senhas não serão \\ armazenadas diretamente no banco, serão \\ criptografadas antes de serem armazenadas \\ como um \textit{hash}.\end{tabular}} & \multicolumn{1}{l|}{Segurança}          &  \\ \cline{1-3}
\multicolumn{1}{|l|}{RNF02}                 & \multicolumn{1}{l|}{\begin{tabular}[c]{@{}l@{}}Comunicação: A comunicação entre as \\ camadas da aplicação deverá ser feita utilizando \\ o protocolo HTTPS, para garantir a segurança \\ no envio dos dados através da rede.\end{tabular}}        & \multicolumn{1}{l|}{Segurança}          &  \\ \cline{1-3}
\multicolumn{1}{|l|}{RNF03}                 & \multicolumn{1}{l|}{\begin{tabular}[c]{@{}l@{}}Responsividade: O sistema deve exibir \\ corretamente os elementos da interface gráfica \\ nos mais variados tamanhos de celulares.\end{tabular}}                                                   & \multicolumn{1}{l|}{Usabilidade}        &  \\ \cline{1-3}
\multicolumn{1}{|l|}{RNF04}                 & \multicolumn{1}{l|}{\begin{tabular}[c]{@{}l@{}}Internacionalização: O sistema deverá suportar \\ dois idiomas (inglês e português) e suportar \\ que futuramente seja possível adicionar outros \\ idiomas.\end{tabular}}                          & \multicolumn{1}{l|}{Usabilidade}        &  \\ \cline{1-3}
\multicolumn{1}{|l|}{RNF05}                 & \multicolumn{1}{l|}{\begin{tabular}[c]{@{}l@{}}Escalabilidade: O sistema deverá ser projetado \\ para garantir que futuras melhoras e expansões \\ sejam possíveis.\end{tabular}}                                                                  & \multicolumn{1}{l|}{Desempenho}         &  \\ \cline{1-3}
\multicolumn{1}{|l|}{RNF06}                 & \multicolumn{1}{l|}{\begin{tabular}[c]{@{}l@{}}Disponibilidade: O sistema deverá estar disponível \\ aos usuários ininterruptamente\end{tabular}}                                                                                                  & \multicolumn{1}{l|}{Disponibilidade}    &  \\ \cline{1-3}
                                            &                                                                                                                                                                                                                                                    &                                         & 
\end{tabular}
\label{reqNaoFuncionais}
\fonte{Os Autores}
\end{quadro}