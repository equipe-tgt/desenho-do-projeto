\section{Viabilidade Financeira}

O projeto de análise de viabilidade financeira consiste em averiguar a garantia de lucro sobre as despesas do projeto. Portanto, nesse projeto será descrito cada processo a fim de fazer essa verificação.

\subsection{Gerenciamento de Custos}

Nesse tópico, serão abordados temas de investimento inicial e de desenvolvimento do projeto, incluindo tópicos de análise de requisitos, desenvolvimento, manutenções e imprevistos.

\subsubsection{Análise de Requisitos e Desenvolvimento}

Para iniciar o projeto, é necessário fazer os primeiros planejamentos, elicitação de requisitos, abstrair e concretizar as primeiras ideias e fazer os primeiros planejamentos (diagramas, cronogramas e documentação). Logo após, o projeto chega na fase de desenvolvimento, onde é começado a se tornar real.

Contudo, o projeto não vai possuir nenhum custo de análise e implementação do sistema, devido ao fato de ser um projeto educacional.

\subsubsection{Manutenções}

Inevitavelmente, manutenções do sistema ocorrerão pós finalização do projeto e estar devidamente funcional em produção. Contudo, os custos de manutenções também não serão cobrados, devido a ser um projeto educacional.

\subsection{Custos de deploy e de Ambiente de Produção}

Nesse tópico, são apresentados os custos de manter o sistema funcional e disponível para os usuários. Desse modo, será feito uma previsão anual de cada plataforma utilizada:

\subsubsection{Frontend} 

Tendo em vista que o projeto é \textit{mobile} voltado para dispositivos Android, será publicado na PlayStore, estimando um valor de 25.00 USD anual. 

\subsubsection{Backend}

Inicialmente gratuito no Amazon EC2, sendo permitido 750h de instâncias por mês durante o período de 12 meses.

A partir do momento que for necessário grande porte, será indicado o plano Sob Demanda do Amazon EC2, que garante viabilidade econômica e estratégica (visto que o preço é calculado a partir do uso). 

Utilizando a calculadora da AWS e optando por um servidor Linux da instância t4g.micro com 1 vCPU e 1GiB, com armazenamento SSD de uso geral de 10GB, será custeado o valor de 7,52 USD mensalmente para operar o mês inteiro.

\subsubsection{Banco de Dados} 

Inicialmente gratuito no Amazon RDS, sendo permitido 750h de instâncias durante o período de 12 meses. O Amazon RDS possui suporte a vários \gls{sgbd}, incluindo o MySQL, que foi o \gls{sgbd} optado para desenvolver a aplicação Lixt.

A partir do momento que for necessário grande porte, será indicado o plano Sob Demanda do Amazon RDS, que garante viabilidade econômica e estratégica (visto que o preço é calculado a partir do uso).

Utilizando a calculadora da AWS e optando por um servidor da instância t2.micro de modelo Single-AZ OnDemand, com armazenamento SSD para cada instância de 10GB, será custeado o valor de 27,74 USD mensalmente para operar o mês inteiro.

\subsection{Medidas de Obtenção de Retorno Financeiro}

Para gerar uma receita positiva a fim de obter lucro, haverá duas formas principais de retorno financeiro:

\begin{itemize}
	\item \underline{Cobrança do aplicativo}: O aplicativo estará disponível gratuitamente na PlayStore, não gerando, portanto, retorno financeiro.
	\item \underline{Propaganda/Recomendação}: Será utilizado mediador de anúncio AdMob (responsável por conectar aplicações e anunciantes), onde o valor varia por visualizações de anúncios e cliques neles. Contudo, no próprio site do admob, é citado um caso no qual houve 300.000 downloads e arrecadava, através do AdMob, 100 USD por dia. \cite{GoogleAdMob}
\end{itemize}

A partir da calculadora do AdMob, estimando-se uma quantidade de 250 visitantes por dia, onde há 2 paginas visualizadas por visitante, sendo que a taxa de cliques em anúncios é 1\% e o custo do clique é 0.35 USD, o valor mensal será de 52.5 USD.

\subsection{Conclusão}

Concluindo que, ao utilizar os servidores de baixo porte detalhados acima, será desembolsado cerca de 35.00 USD por mês. Contudo, o valor calculador para 250 visitantes (estimando o valor com baixo engajamento) com os parâmetros detalhados arrecadará 52.5 USD, conseguindo custear os servidores e ainda garantindo margens de lucro.

Conforme o engajamento na aplicação for aumentando, será revisto os planos dos servidores para atender maiores níveis de requisições.

