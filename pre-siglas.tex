% ---
% inserir lista de abreviaturas e siglas
% ATENCAO o SHARELATEX/OVERLEAF GERA O GLOSSARIO SOMENTE UMA VEZ
% CASO SEJA FEITA ALGUMA ALTERAÇÃO NA LISTA DE SIGLAS É NECESSARIO UTILIZAR A OPÇÃO :
% "Clear Cached Files" DISPONIVEL NA VISUALIZAÇÃO DOS LOGS 
% ---
% https://www.sharelatex.com/learn/Glossaries


\ifdef{\printnoidxglossary}{
    %\printnoidxglossary[type=\acronymtype,title=Lista de abreviaturas e siglas,style=siglas]
\begin{siglas}
	\item[SDK]\hypertarget{s:SDK} {Software Development Kit} ---
  	Kit de desenvolvimento de software. Citado em \ref{sig:SDK}
  \item[HTML]\hypertarget{s:HTML} {HyperText Markup Language} ---
  	Linguagem de Marcação de Hipertexto. Citado em \ref{sig:HTML}
  \item[CSS]\hypertarget{s:CSS} {Cascading Style Sheets} ---
  	Folhas de Estilo em Cascata. Citado em \ref{sig:CSS}
  \item[API]\hypertarget{s:API}{Application Programming Interface} ---
    Interface de progragramação de Aplicação. Citado em \ref{sig:API}
  \item[HTTPS]\hypertarget{s:http}{Hypertext Transfer Protocol} ---
    Protocolo seguro de transferência de hypertexto. Citado em
    \ref{sig:https}
  \item[REST]\hypertarget{s:rest}Representational State Trasfer ---
    Transferência de Representação deEstado: modelo de transferência
    de dados no qual o estado de um objeto é serializado e transferido
    entre aplicações. Citado em \ref{sig:rest}
   \item[ORM]\hypertarget{s:ORM}Object–relational mapping --- Mapeamento objeto-relacional. Citado em 		\ref{sig:ORM}
   \item[MVP]\hypertarget{s:mvp}{\textit{Minimum Viable Product}} --- 
  	Mínimo Produto Viável. Citado em \ref{sig:mvp}
   \item[LGPD]\hypertarget{s:lgpd}{Lei Geral de Proteção de Dados}. --- 
  	Citado em \ref{sig:lgpd}
   	
\end{siglas}    
    \cleardoublepage
}{}


