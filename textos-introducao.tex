\chapter{Introdução}

\label{sec:contextualizao}
\section{Contextualização}
Ir às compras é uma atividade essencial no cotidiano das pessoas. Tanto por uma questão de sobrevivência, envolvendo compra de alimentos, medicamentos, produtos de higiene pessoal e da casa, quanto por uma questão de lazer.

Quando se trata de compras essenciais, ou seja, aquelas nas quais os produtos adquiridos tem o papel de reabastecer funções de necessidades básicas, as compras de varejo alimentar não podem ser desconsideradas, devido sua importância e frequência. Conforme a pesquisa “Tendências e Comportamento do Consumidor” \cite{APAS}, realizada em 2018 pela Associação Paulista de Supermercados (APAS), 89\% da população com 16 anos ou mais costuma ir ao supermercado mesmo que ocasionalmente. Um consumidor, em média, faz quatro visitas ao supermercado por mês e um em cada três frequentadores costuma ir ao supermercado pelo menos uma vez por semana. Além disso, os supermercados foram apontados como o estabelecimento favorito de varejo alimentar por 53\% dos participantes, seguido pelo hipermercado, com 13\%.

Para melhor entendimento dos resultados apresentados e que serão apresentados ao decorrer do documento, é necessária a definição das diversas categorias de estabelecimentos de compra de varejo alimentar com o formato autosserviço, que se refere à estabelecimentos em que o consumidor compra o produto sem a necessidade de haver a intermediação de um funcionário da loja, antes de realizar o \textit{check-out}. Esses estabelecimentos podem ser categorizados como:
\begin{itemize}
\item Supermercado: Se refere à loja alimentar caracterizada pela disposição de produtos no formato autosserviço, separados por quatro áreas básicas (perecíveis, mercearia, limpeza doméstica e bebidas) e pela existência de pelo menos dois \textit{check-outs} na saída. Possui de 1.500 a 5.000 itens disponíveis.
\item Hipermercado: Se refere à loja alimentar do tipo autosserviço, que oferece vendas de mercearia, bazar, perecíveis, têxteis e eletrodomésticos. Possuem mais de 5.000 itens em exposição e mais de 40 \textit{check-outs} espalhados no estabelecimento.
\item Lojas com apenas 1 \textit{check-out}: Pequenas lojas de varejo alimentar, como mercearias, lojas de bairro e conveniência. Apesar que, no caso das lojas de conveniência, pode ser encontrados estabelecimentos com até três \textit{check-outs}.
\end{itemize}

Além de estabelecimentos de varejo alimentar com o formato autosserviço, também são encontradas lojas do setor de hortifrúti, como feiras livres e lojas de bairro focadas exclusivamente nesse tipo de alimento. Essas lojas podem ser chamadas de mercados independentes, pois são operadas por empresas que possuem até 4 lojas que operam sobre a mesma marca. De forma contrária, uma rede de mercados ou supermercados é formada por no mínimo 5 lojas que operam sobre a mesma marca.

Entretanto, alguns hábitos de consumo foram modificados devido à pandemia da COVID-19, declarada nos primeiros meses de 2020. O isolamento social, considerado uma importante medida de prevenção ao coronavírus no Brasil e no restante do mundo, fez com que os consumidores brasileiros realizassem menos visitas aos estabelecimentos de varejo alimentar, mas levado cada vez mais produtos para casa, segundo o levantamento de dados realizado pela Dotz \cite{Dotz}. Outra pesquisa, realizada pela Asserj \cite{Asserj}, apontou que 84\% dos entrevistados vão ao supermercado apenas uma vez por mês, enquanto 10\% fazem compras duas vezes por mês e 6\% restante afirma fazer compras semanalmente. Os dados mostrados refletem a mudança de hábito dos consumidores, que antes compravam gradualmente, ao longo do mês, mas que agora, passou a concentrar as compras tudo em um único dia, em busca de evitar exposição ao coronavírus, e economizar.

Além da mudança de hábitos relacionados à frequência e quantidade de produtos em cada compra, as compras de produtos básicos para cozinhar em casa cresceram, e a busca por compras em supermercados, hipermercados e atacarejos também, de modo a criar estoques de alimentos nas casas, segundo pesquisa realizada pela Nielsen Brasil \cite{NielsenBrasil}.

De certa maneira, pode-se dizer que, independente das condições externas, as compras em estabelecimentos de varejo alimentar são muito presentes no cotidiano das famílias brasileiras. Por isso, devem ter sua devida atenção e gerenciamento, principalmente por questões financeiras, e de organização, a fim de alcançar o objetivo de comprar mais produtos de maior qualidade gastando menos.

\label{sec:problematizacao}
\section{Problematização}
Por ser algo tão presente na vida de todas as famílias do país, independentemente da frequência das visitas aos estabelecimentos e a quantidade de produtos nos quais são comprados, a falta de planejamento no momento de realizar uma nova compra pode acarretar problemas financeiros.

Em uma pesquisa realizada em 2015 pela empresa Serviço de Proteção ao Crédito (SPC Brasil) \cite{SPC}, foi apontado que 33\% das compras feitas por impulso são de supermercado ou estabelecimentos de varejo alimentar. As razões principais que os participantes da pesquisa compartilharam foram em relação ao produto estar mais barato, devido a uma promoção, ou por falta de planejamento.

Além das compras por impulso, que geram valores mais altos no \textit{check-out}, outro problema gerado pela falta de planejamento é o destino dos produtos comprados. Segundo o Instituto Akatu \cite{Akatu}, cerca de 30\% das compras de mercado vão para o lixo.

Tendo em vista as consequências que a falta de planejamento nas compras pode gerar, como altos gastos e desperdício de alimentos e dinheiro, é possível concluir que as pessoas não saibam administrar e controlar seus gastos em supermercados, mesmo existindo meios de fazê-lo.

Atualmente, as maneiras tradicionais de um consumidor de fazer pesquisas de preço e controlar seus gastos é através dos panfletos oferecidos pelos estabelecimentos, nos quais mostram ofertas e descontos, e através do cupom fiscal. De acordo com a Pesquisa Tendências do Consumidor de 2018 \cite{PesquisaAPAS}, 69\% dos participantes declararam que realizam pesquisas de preço antes de comprarem determinado produto, através das seguintes formas:
\begin{itemize}
\item Indo ao supermercado (59\%).
\item Consultando ofertas em jornais e folhetos (50\%).
\item Buscas na Internet (12\%).
\end{itemize}

No momento de controlar os gastos e administrá-los de maneira detalhista, a única forma é através do cupom fiscal, um documento frágil que contém todos os produtos comprados e seus respectivos preços, além dos impostos cobrados, data e hora da compra e o nome do estabelecimento. Sua fragilidade é devido a sua composição, o que ocasiona o desaparecimento das informações ao decorrer do tempo. Ao depender do cupom fiscal para analisar e atualizar os gastos, ocorrem alguns dos seguintes problemas:
\begin{itemize}
\item Só é possível fazer uma análise de gastos após a compra, sendo que o consumidor teria grandes benefícios ao analisar os gastos e produtos durante a compra.
\item Dificuldade de lembrar e guardar os preços dos alimentos e produtos durante a compra, para futura análise que auxilie o comprador a fazer compras mais econômicas e de qualidade.
\item É necessário repassar manualmente todos os dados de cada cupom fiscal para um documento centralizado, caso o consumidor queira manter um histórico de compra ou fazer uma análise geral.
\item Dificuldade em analisar preços de diferentes estabelecimentos de modo a decidir qual o melhor custo-benefício julgado pelo comprador, já que essa informação vai provavelmente constar em algum meio não centralizado, ou seja, o consumidor irá perder a informação rapidamente.
\item Dificuldade em analisar e calcular quais as categorias de produtos que possuem maior gasto, menor gasto, ou uma média, baseados em determinado período, que visa facilitar a descoberta de quais produtos são mais comprados, quais são mais caros, de modo a auxiliar o usuário em futuras compras.
\item Dificuldade de gerenciar compras colaborativas, cada participante não sabe exatamente o que será comprado ou quais itens serão de sua responsabilidade, a não ser que essas informações sejam compartilhadas através de meios que a ação manual seria necessária, como mandar uma mensagem através de um aplicativo.
\end{itemize}

\label{sec:justificativa}
\section{Justificativa}
Para o total controle e gerenciamento das compras, seria necessário utilizar métodos para a pesquisa de preço, e a análise do cupom fiscal. Apesar de ser possível utilizar a busca na Internet para realizar a pesquisa de preço, essa forma é pouco utilizada entre as pessoas, além de as informações não serem centralizadas, ocasionando na demora para realizar a pesquisa. Além disso, ambos métodos são manuais, e não há muitas possibilidades de automatizá-los. A falta de automatização gera maior dificuldade para as pessoas realizarem seus planejamentos, o que acaba ocasionando em gastos cada vez mais altos em cada compra realizada, sendo cada vez mais difícil de entender e analisar os motivos que fazem o bolso ficar mais pesado no final do mês.

O agravamento dos problemas financeiros podem trazer consequências graves, fora ser um tópico presente nas famílias brasileiras. Conforme estudos realizados pela SPC Brasil em 2020, foi revelado que a meta principal do brasileiro é guardar dinheiro \cite{MetaFinanceiraSPC}, e que apenas 1 a cada 10 brasileiros tem renda suficiente para pagar as despesas de início de ano \cite{DespesaSPC}. Além do mais, problemas financeiros podem gerar problemas psicológicos e de saúde, como foi levantado pelo estudo realizado pela The Employer’s Guide to Financial Wellness 2019 \cite{Endividados}, que apontam que pessoas endividadas têm 4 vezes mais chances de desenvolver depressão, e 8 vezes mais chances de desenvolver insônia.

Considerando os problemas citados acima, o gerenciamento e administração de compras é extremamente essencial no contexto vivido por tantos brasileiros, tanto por uma questão financeira, de administrar as contas, sobreviver, e construir sonhos, como também de saúde.

\label{sec:objetivos}
\section{Objetivos}
Para solucionar todos os problemas citados e no que eles acarretam, entregando autonomia, controle de compras para os consumidores de maneira intuitiva e fácil, em que todas as informações que eles registrem sejam armazenadas em um local centralizado, criamos o Lixt, uma solução que facilite a vida do consumidor que reside no Brasil, para o gerenciamento e controle de suas compras através de criação e edição de listas de compras compartilhadas ou individuais. A solução será focada em facilitar as compras alimentícias, feitas em supermercados, feiras, lojas de conveniência, e até mesmo em aplicativos de alimentos, como IFood, Uber Eats, Rappi, e entre outros.

Além disso, o Lixt se propõe em oferecer análises de compras por determinado período escolhido pelo usuário, separar produtos em categorias e até mesmo apresentar análises quanto a variação de preço entre estabelecimentos, que poderão ser adicionados de acordo com a preferência do usuário, além de outras funcionalidades que estarão melhores descritas na seção de Escopo. Dessa maneira, o planejamento de compras alimentares torna-se automatizada e substituí formas manuais de gerenciamento de gastos.

\label{sec:analiseconcorrencia}
\section{Análise da Concorrência}
Auditamos soluções que existem atualmente no mercado e, ao verificar as aplicações existentes, conclui-se que há intersecções nas funções dentre os aplicativos analisados. As funções mais básicas, como gerenciamento de itens e gerenciamento de listas, estão presentes em todos, tendo em vista que são essenciais em qualquer aplicativo de
lista. Outras funções básicas que deveriam ser incluídas em qualquer aplicação de lista, como gerenciamento de categorias e
compartilhamento de listas, não estão presentes em todos os
aplicativos analisados.

Contudo, as divergências ficam claras quando analisamos o mecanismo das aplicações, entre elas destacam-se o \textit{Mealime} e o \textit{Cozi Family Organizer} que, apesar de serem voltados para as compras, cumprem também outras funcionalidades. O \textit{Mealime}, cujo foco é o planejamento de refeições, e o \textit{Cozi Family Organizer}, cujo foco é o planejamento familiar, deixam a desejar nas funções relacionadas às compras.

Entre os outros aplicativos analisados, é perceptível que não possuem todas as funcionalidades propostas nesse documento, principalmente quando se trata de compartilhamento de listas, dado	 que cada \textit{software} lida de modo diferente diante dessa \textit{feature}. O \textit{SoftList}, por exemplo, permite o compartilhamento de lista, porém não pode  ser gerenciada por mais de um usuário, sendo apenas importada para o usuário no qual a lista está sendo compartilhada.

Ao analisar os aplicativos mais populares da categoria, constatamos que o \textit{Out Of Milk}, \textit{Bring!} e o \textit{OurGroceries}, sendo destaques na área, não se propõem a exibir análise estatística das compras do usuário e nem manter um histórico do que foi comprado. A tabela \ref{tbl:concorrentes} permite visualizar melhor as diferenças entre os concorrentes.

\label{tbl:concorrentes}
\begin{table}[h]
\centering
  \resizebox{\columnwidth}{!}{%
    \begin{tabular}{lccccccc}
      \hline
      & Cozi Family Organizer & OurGroceries & Softlist & Out of Milk & Mealime
      & Bring & Lixt\\
      \hline
      Login/Cadastro & x & x & x & x & x & x & x\\
      \hline
      Categorias &  & x & x & x &  & x & x\\
      \hline
      Compartilhamento de listas & x &  & x & x &  & x & x\\
      \hline
      Atribuição de itens &  &  &  &  &  &  & x\\
      \hline
      Gerenciamento de compras &  &  & x &  &  &  & x\\
      \hline
      Historico de compras &  &  & x &  &  &  & x\\
      \hline
      Análise de compras &  &  & x &  &  &  & x\\
      \hline
      Calculadora &  &  & x & x &  & x & x\\
      \hline
      Comentários &  &  &  &  &  &  & x\\
      \hline
    \end{tabular}
  }
  \caption{Tabela \ref{tbl:concorrentes}: Uma comparação dos aplicativos concorrentes.}
\end{table}
%%% Local Variables:
%%% mode: latex
%%% TeX-master: "../desenho"
%%% End:
